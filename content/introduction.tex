% Main body of manuscript
\section{Introduction}
\firstword{L}{he}% Capital letter in first word
Academic Journal on Computing, Engineering and Applied Mathematics is a publication of the Federal University of Tocantins, Palmas/TO, Brazil. It provides coverage of significant work on principles  of computer science, engineering and applied mathematics. It is published in electronic format with a quarterly frequency.

This journal accepts papers written in English or Portuguese. The default language is English, but you can change that by replacing the \texttt{eng} option to \texttt{por} at the beginning of this document. The papers contain: title, abstract and keywords must be in English. The body of the text should preferably be in English or Portuguese. In the case of a Portuguese text, it must have a Portuguese and English title, abstract, and keywords with the commands \verb|\titulo{}|, \verb|\resumo{}|, and \verb|\palavraschave{}|, respectively.

AJCEAM also has four types of documents: \textit{Research Papers} -- material based on an original and unclear scientific hypothesis, which is validated or not through experimentation or theoretical models, based on the scientific method, with adequate statistical planning and discussion with scientific argumentation (minimum of five and maximum of ten pages); \textit{Brief Communication} -- short and / or quick communication of results of original and unclear scientific research, which is validated or not through experimentation or theoretical models, based on the scientific method, with adequate statistical planning and brief discussion with scientific argument (minimum of three and maximum of four pages).; \textit{Technical Report} -- a document of a technical nature to describe procedures for the elaboration of experiments, use cases in tools for the specific use of a certain area of research or performance, demonstration or proof of concepts that can help or be used in the development of research (minimum of five and maximum of ten pages); and \textit{Special Session} -- a document that presents the performance of a systematic review, study mapping, survey (a study that tries to answer research questions based on the literature in a broader way than a systematic review or study mapping) or represents a vision or trend based on the literature, expressing the researcher's opinion (minimum of three and maximum of ten pages).

A Research Paper must contain: title, abstract, keywords, introduction, problem statement, solution, results, conclusions and bibliography, if an article does not agree with this, it must be rejected. It is important to note that the text should not necessarily present these topics, however, these elements must be present in the text.

The paper must clearly justify its motivation and there must be some innovative element in it. This contribution can be theoretical and/or experimental and must be demonstrated in relation to previous works duly cited. Articles in which only procedures are implemented and in which the innovation is not clear should be rejected.

In addition to its research quality, an article may be rejected for poor presentation (including images, spelling, grammar); as well as a poor originality in relation to previously published works. If you have questions about the originality of an article, please contact the Associate Editor directly, this may help you to assess the percentage of coincidence with previous work (Plagiarism Percent Match). Percentages greater than 40\% in relation to other research articles in any language are not acceptable.

This document is based on the IEEE template used for most of its publications and conferences. However, some changes have been made, e.g., the top and left margins have been set to 2cm as well as the right and bottom margins have been set to 1.5cm. The column width has been set to 8.5cm, with a separation of 0.5cm between them. In addition, the page header has been moved to the right side that indicates the article code (which will be completed in the final editing and publishing process). Only the page number is included in the footer.

The elementary definitions of style are Times or Times New Roman font for all parts of the document, size 20pt for the title (in the option of papers in Portuguese, it must be included a title in English), 12pt for the authors, 9pt and italic for the line of the Institution the authors belong to, 9pt for the abstract and keywords (in the option of papers in Portuguese, it must be included an abstract and keywords in English), 10pt for the normal text and equations, and 12pt for section titles, 11pt for the level 1 title, italic for the level 2 and 3 titles, 9pt for the titles epigraphs of the figures, tables, and references (all these definitions are already fixed in the provided class file). Use only \emph{italics} to highlight a term. Despite all these details and how many others that could be given, it is recommended to write your article by copying, pasting and replacing text from this document. This is the easiest and safest way to respect the defined styles. Please do not redefine any element of the style (typography, spaces between texts, margins, or other measures defined in the class file).

The general structure expected for this article includes sections such as Introduction, Materials, Methods, Results, Discussion, Conclusions, Acknowledgments and References. These titles can be combined in pairs in the same section and the titles Future Works and Acknowledgments are totally optional. It is common for the Methods section to have another title more related to the original contribution of the article, but the remaining sections are presented with the titles listed above. If there are demonstrations or other extensive mathematical developments, it is recommended to group them in Appendices before bibliographic references.

More details about the sections of the document and the formats for inserting different types of objects, such as equations, figures, and tables, will be given below.